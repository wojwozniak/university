\documentclass{article}
\usepackage{graphicx}
\usepackage[polish]{babel}
\usepackage[utf8]{inputenc}
\usepackage[T1]{fontenc}
\usepackage[top=0.5in]{geometry}
\usepackage{titling}
\usepackage{systeme}
\usepackage{setspace}
\usepackage{amsmath}
\onehalfspacing
\begin{document}

\title{Zadanie egzaminacyjne 2\\
    Rachunek prawdopodobieństwa i statystyka}
\author{Wojciech Woźniak}
\date{}
\maketitle

\section*{Opis zadania}
Mój numer indeksu kończy się $44$, więc n = 5; m = 5.
\\
Na trójkącie o wierzchołkach $(0,0), (5,0), (5,5)$ zmienna $(X,Y)$ ma stałą gęstość $f(x,y) = C$. Mamy wyznaczyć gęstość zmiennej $T = X + 2Y$.

\section{Rysunek poglądowy, wyznaczenie wartości C}

\noindent
\begin{minipage}[t]{0.6\textwidth}
    \vspace*{-4.5cm}
    Rysunek poglądowy początkowego trójkąta.\\
    Trójkąt wyznaczony jest prostymi y = x, y = 0 i x = 5.\\
    Pole tego trójkąta to $\frac{25}{2}$, więc C we wzorze na gęstość to $\frac{2}{25}$.\\
    $f(x,y)$ = $\frac{2}{25}$
\end{minipage}
\hfill
\begin{minipage}[t]{0.35\textwidth}
    \includegraphics[width=\textwidth]{skrt.png}
\end{minipage}

\section{Przejście do zmiennej (S, T), odwrócenie równania i obliczenie modułu Jakobianu}

\begin{flalign*}
    \systeme{
        T = X + 2Y \quad \quad \Rightarrow \quad \quad X = T - 2Y,
        S = Y
    } &  &
\end{flalign*}
\begin{flalign*}
    \systeme{
        X = T-2S,
        Y = S
    } &  &
\end{flalign*}
\begin{flalign*}
    |J| = \det \left( \begin{bmatrix}
                          \frac{\partial X}{\partial T} & \frac{\partial X}{\partial S} \\
                          \frac{\partial Y}{\partial T} & \frac{\partial Y}{\partial S}
                      \end{bmatrix} \right)
    = \det \left( \begin{bmatrix}
                      1 & -2 \\
                      0 & 1
                  \end{bmatrix} \right)
    = 1 &  &
\end{flalign*}

\pagebreak
\onehalfspacing

\section{Całka nieoznaczona z gęstości (wzór z s)}
$g(t,s) = f_{X,Y}(x(t,s), y(t,s)) \times |J| = \frac{2}{25} * 1 = \frac{2}{25}$\\\\
$\int g(t,s) ds = \frac{2}{25} s + C$

\section{Ograniczenia t}
$t = x + 2y$ \quad \quad \quad $y = x$ \quad \quad \quad $t = x + 2x = 3x$\\\\
$x \in [0,5] \rightarrow t \in [0, 15]$

\section{Ograniczenia s przy zmieniającym się t}
$0 \leq x \leq 5$ \quad \quad \quad \quad $0 \leq y \leq x$\\
$0 \leq t-2s \leq 5$ \quad \quad $0 \leq s \leq t-2s$\\\\
Rozwiązując te dwie podwójne nierówności otrzymujemy:\\\\
max(0, $\frac{T-5}{2}$) $\leq s \leq \frac{T}{3}$

\section{Wyznaczenie przedziałów dla funkcji gęstości}
Proste s = 0 i $s = \frac{t-5}{2}$ przecinają się gdy $t = 5$ (najpierw większa
jest pierwsza, potem druga). Mamy więc:\\ - $t \in [0,5] \rightarrow s \in [0,
        \frac{t}{3}]$\\ - $t \in [5, 15] \rightarrow s \in [\frac{t-5}{2},
        \frac{t}{3}]$\\ \noindent \\ \includegraphics[width=1\linewidth]{Bez
    tytułu2.png}\\ Rysunek poglądowy, zauważmy że $\frac{t-5}{2} \leq \frac{t}{3}$
dla $t \in [0, 15]$

\section{Wyznaczenie wzoru funkcji gęstości}
Liczymy całki z g(t,s) ds na odpowiednich przedziałach. (Pierwsza od 0 do
$\frac{t}{3}$, druga od $\frac{t-5}{2}$ do $\frac{t}{3}$). Otrzymujemy:\\\\ $
    g_1(t) =
    \begin{cases}
        \frac{2}{75}t     & \text{dla } t \in [0,5]  \\[10pt]
        \frac{15 - t}{75} & \text{dla } t \in [5,15]
    \end{cases}
$

\section{Sprawdzenie wyniku}
\noindent

$$
    \int_0^5 \frac{2}{75}t \, dt + \int_5^{15} \frac{15-t}{75} \, dt = \frac{1}{3} + \frac{2}{3} = 1,
    \quad \text{więc } g_1(t) \text{ jest gęstością.}
$$

\end{document}